\section{Wykorzystane narzędzia psychologiczne}
\subsection{Job Satisfaction Survey - Paul E. Spector}
\emph{Job Satisfaction Survey} jest profesjonalnym testem psychologicznym składającym się z 36 stwierdzeń: po 4 stwierdzenia na każdy z przyjętych 9 wymiarów satysfakcji z pracy (m.in.: związanych z nastawieniem pracownika do pracy czy organizacją pracy). Wymiary zostały dobrane na podstawie dostępnej literatury na ten temat. Twórcom zależało, aby skale odzwierciedlały najważniejsze aspekty satysfakcji z pracy. Przy czym skale miały być jasno
rozróżnialne (jasne granice). Prezentowane stwierdzenia są przygotowane w obu kierunkach: pozytywnym oraz negatywnym, co ma istotne znaczenie przy liczeniu wyników. 

Zastosowane skale to:
\begin{itemize}
\item Płaca -- płaca za wykonaną pracę
\item Awans -- możliwości awansu
\item Nadzór -- ocena bezpośredniego kierownictwa
\item Świadczenia pracownicze -- premie i dodatki (finansowe i niefinansowe)
\item Warunkowe nagrody -- docenienie, uznanie oraz nagrody za dobrą pracę
\item Organizacja pracy -- obowiązujące zasady i procedury, polityka w firmie
\item Współpracownicy -- osoby, z którymi współpracujesz na co dzień
\item Praca -- natura oraz specyfika wykonywanej pracy
\item Komunikacja - komunikacja w ramach organizacji
\end{itemize}

Pierowtnie \emph{JSS} został stworzony dla pracowników sektora uslugowego, jednak może być stosowany do wszystkich typów organizacji. Obecnie istnieją normy dla wielu typów organizacji oraz pracy, dostępnych na stronie TODO.

Do każdego z 36 stwierdzeń responded ustosunkowuje się wskazując stopień zgodności na sześciostopniowej skali nominalnej:
\begin{itemize}
\item bardzo się zgadzam
\item umiarkowanie się nie zgadzam
\item minimalnie się nie zgadzam
\item minimalnie się zgadzam
\item umiarkowanie się zgadzam
\item bardzo się nie zgadzam
\end{itemize}
Skala ta zwyczajowo jest transofrmowana na skalę liczbową od 1 (,,bardzo się nie zgadzam") do 6 (,,bardzo się zgadzam"). Dzięki takiemu przekodowaniu możliwe są dalsze wyliczenia statystyk. Oczywiście dla stwierdzeń negatywnych wymagane jest dodatkowe przekodowanie w celu uzyskania sumarycznego wyniku dla pozytywnej notacji. Przekodowanie to jest odwróceniem skali liczbowej, czyli:
\begin{itemize}
\item $ 1 -> 6 $
\item $ 2 -> 5 $
\item $ 3 -> 4 $
\item $ 4 -> 3 $
\item $ 5 -> 2 $
\item $ 6 -> 1 $
\end{itemize}

Końcowa satysfkacja pracownika wyliczona za pomocą \emph{JSS} reprezentowana jest w postaci liczbowej na skali ciągłej rozciągającej się od 36 włącznie do 216 włącznie, czyli od zdecydowanego niezadowolenia do zdecydowanego zadowolenia z pracy. Nie ma z góry określonych liczbowych wartości granicznych pomiędzy zadowoleniem, a niezadowoleniem z pracy. Ze względu na konieczność oceny, przyjęto 2 podejśćia:
\begin{itemize}
\item podejście normatywne -- porównanie z normami dla danej populacji,
\item podejście absolutne -- podzielenie skali na 3 przedziały liczbowe, oznaczające po kolei niezadowolenie z pracy, uczucia ambwiwaletne w stosunku do pracy oraz niezadowolenie z pracy.
\end{itemize}

\subsubsection{Podejście normatywne}
Podejście normatywne porównuje badaną próbę osób z normami dla tej samej populacji. Część norm jest ogólniedostępna, w szczególoności dla angielskojęzycznej populacji, na strone twórcy \emph{JSS} TODO. Wykorzystujemy test statystyczny t-studenta podczas porównania stwierdzamy, czy istnieje znacząca statystycznie różnica między badaną grupą, a populacją. W wyniku możemy stwierdzić, że badani są tak samo, bardziej lub mniej zadowoleni z pracy niż przeciętne osoby. Niestety podejście to
ma trzy poważne wady:
\begin{enumerate}
\item Obecnie mały procent typów organizacji i zawodów posiada normy. Wyklucza to możliwość zastosowania tego podejśćia w dużej ilości orgaqnizacji oraz ogranicza możliwości porównania.
\item Normy prezentowane na stronie \emph{JSS} zazwyczaj nie są reprezentatywne, a jedynie stworzone są na podstawie próbek podesłanych autorówi \emph{JSS} przez innych naukowców. W związku z tym nie tworzą ważnych naukowo norm i wyciąganie wniosków na ich podstawie powinno być ostrożne.
\item Większość norm została zebrana z terenów Ameryki Północnej. Satysfakcja z pracy różni się ze względu na kraj, a tym bardziej kulturę, dlatego wykorzystanie obecnych norm w celach innych niż porównanie z krajami Ameryki Północnej jest niemiarodajne.
\end{enumerate}
 
\subsubsection{Podejście absolutne}

Ze względu na wymienione wady podejścia normatywnego, wyznaczono także podejście absolutne bazujące na podziale domeny wyników. Każde ze stwierdzeń jest oceniane na skali od 1 (zdecydowanie się nie zgadzam) do 6 (zdecydowanie się zgadzam). Po uwzględnieniu odwrócenia skali dla stwierdzeń negatywnych, można przedstawić domenę:
\begin{itemize}
\item dla każdej z podskal -- [4, 24],
\item dla ogólnego wyniku -- [36, 216],
\end{itemize}
Domeny te zostały podzielone ze względu na trzy wyróżnione stany:
\begin{description}
\item[niezadowolenie z pracy] \hfill \\
dla każdej z podskal [4, 12), \hfill \\ dla sumarycznego wyniku [36, 108),
\item[ambwiwalentny stosunek do pracy] \hfill \\
dla każdej z podskal [12, 16), \hfill \\ dla sumarycznego wyniku [108, 144],
\item[zadowlenie z pracy] \hfill \\
dla każdej z podskal [16, 24], \hfill \\ dla sumarycznego wyniku [144, 216].
\end{description}

Do powyższych obliczeń można także wykorzystać średnią wartość dla uzyskanych wyników. Wówczas odpowiednio:
\begin{description}
\item[niezadowolenie z pracy] \hfill \\ poniżej 3 ($śr. <3$),
\item[ambwiwalentny stosunek do pracy] \hfill \\ pomiędzy 3 włącznie, a 4 wyłącznie ($ 3<= śr. < 4 $)
\item[zadowlenie z pracy] \hfill \\ powyżej 4 włącznie ($ śr. >= 4 $).
\end{description}

Podejściu to także unika jasnego podziału między zadowoleniem, a niezadowleniem wprowadzając przdział związany z mieszanymi uczuciami odnośnie pracy.

\subsubsection{Powiązania z różnymi aspektami pracy}
Opracowując \emph{JSS} wykonano także inne znane testy psychologiczne (np.: Organizational Commitment Questionnaire -- Mowday, Steers \& Porter, 1979 lub Leader Behavior Leader Behaviour Description Questionnaire -Stogdill, 1963) , aby sprawdzić powiązania wyników \emph{JSS} z innymi aspektami pracy. Najsilniejsza korelacja wyszła z odbiorem pracy i przełożonych, intencjami odejścia oraz zaangażowaniem organizacyjnym.
Podczas badań nad \emph{JSS}em sprawdzono jak wyniki testu korelują z innymi aspektami pracy
\subsection{Ultrecht Work Engagement Scale}
