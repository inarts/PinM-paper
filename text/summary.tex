\chapter{Podsumowanie}

Badana grupa (patrz rozdział \ref{sec:group}) posiada kilka interesujących cech charakterystycznych:
\begin{itemize}
  \item 84,93\% respondentów to mężczyźni,
  \item ok. 80\% osób to osoby w wieku 30 lat i poniżej,
  \item 80,82\% osób to osoby z co najwyżej czteroletnim doświadczeniem zawodowym,
  \item 78,05\% osób pracuje jako specjaliści różnego szczebla,
  \item ok. 75\% osób pracuje na obecnym stanowisku co najwyżej 3 lata (w tym 45,21\% ogółu całej grupy wpisało dokładnie 1 rok),
  \item 93,15\% osób pracuje na cały etat,
  \item 78,08\% osób pracuje w ramach umowy o pracę,
  \item 78,08\% osób pracuje w firmach powyżej 50 pracowników.
\end{itemize}
Jak widzimy, prawdopodobnie grupa ta nie jest reprezentatywna dla wszystkich pracowników sektora IT. Jednak można zawęzić jej charakterystykę na podstawie powyższych danych oraz porównać wyniki z teorią na temat wpływu różnych cech i vice versa (co zostało zrobione w rozdziale \ref{sec:hyp-ver}). Niestety nie jesteśmy w stanie odnieść się do doświadczenia zawodowego; mamy młodą grupę o krótkim stażu zawodowym. Podobna sytuacja jest ze stanowiskami kierowniczymi, na ponad 70 osób badanych, tylko 10 posiada stanowisko tego typu.

Na podstawie przeprowadzonych badań możemy stwierdzić, że około połowa badanych jest zadowolona ze swojej pracy (wg podejścia absolutnego, patrz rozdział \ref{sec:tests-sat-abs}). Przy czym zdecydowana większość badanych jest zadowolona ze swoich współpracowników (wymiar z najlepszymi wynikiem, ok. 90\% usatysfakcjonowanych pracowników) i przełożonych. Z drugiej strony połowa osób nie jest zadowolona z możliwości awansu w pracy (respondenci są
najmniej zadowoleni z tego wymiaru). Co więcej, ok. 90\% osób jest zadowolonych z zadań jakie wykonują (test \emph{JSS}, wymiar \textit{charakter pracy}) co może wskazywać na dużą odpowiedniość pracy oraz wysokie możliwości wykorzystania w niej swoich umiejętności. Warto by było zbadać dokładniej przyczyny powyższych wyników. Obecnie możemy tylko stawiać hipotezy na podstawie różnych teorii odnośnie satysfakcji z pracy.  Dodatkowo okazało się, że respondenci są bardziej usatysfakcjonowani różnymi aspektami swojej pracy niż przeciętni Polacy.

Kolejnym badanym konstruktem jest zaangażowanie w pracę. Zgodnie z teorią na ten temat, zaangażowanie nie jest ciągłym stanem, dlatego miarą jest częstotliwość występowania (w przeciwieństwie do satysfakcji). Patrząc na wyniki, dwie trzecie badanych odczuwa wigor i absorpcję w pracy od kilku razy w miesiącu do raza w tygodniu (czyli maksymalnie 5 razy w miesiącu). Z identyczną częstotliwością połowa badanych odczuwa dumę ze swojej pracy i widzi w niej sens. Ponadto wysoka korelacja między wymiarami wskazuje albo na ich wysokie semantyczne podobieństwo dla badanych lub bardzo podobne przyczyny powstawania tych emocji. Podsumowując, jak można było się domyślić z wyników na każdym z wymiarów, około dwie
trzecie respondentów odczuwa zaangażowanie w pracę od raza do pięciu razy w miesiącu. Co więcej, jedna czwarta grupy, ma podobne emocje częściej (co najmniej raz w miesiącu). Warto podkreślić, że pomimo posiadania cech charakterystycznych, badana grupa nie odbiega pod tym względem od ogółu Polaków. W związku z tym nasuwa się pytanie, dlaczego? Odpowiedź na nie mogłaby być interesująca, jednak nie jest to tematem niniejszej pracy. 

Na koniec należy wspomnieć o wynikach badania zależności między satysfakcją z pracy, a zaangażowaniem w pracę. Jak się okazało, istnieje tylko średniej mocy zależność (0,42 przy istotności statystycznej 0,05) między tymi konstruktami. Przyglądając się korelacjom na poszczególnych wymiarach, najsilniejsza zależność istnieje między \textit{charakterem pracy} z testu \emph{JSS}, a poszczególnymi wymiarami zaangażowania. Natomiast zaskakujący jest całkowity brak zależności między
\textit{organizacją pracy} z testu \emph{JSS}, a zaangażowaniem.

  Jak widać, badana grupa wypada tak samo (konstrukt zaangażowania), jak nie lepiej (konstrukt satysfakcji) niż ogół Polaków. Także patrząc na konkretne wartości miar (podejście absolutne dla testu \emph{JSS} i częstotliwościowe dla testu \emph{UWES}) można być zadowolonym. Podsumowując jednym zdaniem, dla badanych różne aspekty pracy spełniają ich oczekiwania (co widać po satysfakcji) oraz praca pozwala im wyrazić się w wykonywanych zadaniach (patrz zaangażowanie).
