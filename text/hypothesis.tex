\section{Problemy badawcze oraz stawiane hipotezy}
\subsection{Zależność między satysfakcją, a zaangażowaniem}
\label{sec:hypothesis-relation}
Intuicyjnym wydaje się istnienie zależności między satysfakcją z pracy, a zaangażowaniem pracowników w wykonywane zadania. Skoro satysfakcja wskazuje na poziom zadowolenia z różnych aspektów pracy (warunków, komunikacji, pensji, sposobu zarządzania, itp.), to takie środowisko pracy, które jest satysfakcjonujące powinno sprzyjać także wysokiemu zaangażowaniu. 

Co więcej wybrane wymiary satysfakcji możemy spróbować teoretycznie przełożyć na wymiary zaangażowania. Skoro warunki pracy (satysfakcja, wymiar \textit{warunki pracy}) są odpowiednie, oznacza to, że jest bardzo mało czynników, które powinny nam przeszkadzać w skupieniu
się całkowitym nad zadaniami (wymiar \textit{absorpcji}). Natomiast jeżeli przełożeni dbają o swoich pracowników (satysfakcja, wymiar \textit{nadzór}) oraz cele w firmie są jasno stawiane (satysfakcja, wymiar \textit{komunikacja}), to pracownicy powinny odczuwać sensowność wykonywanych zadań oraz dumę z pracy dla danej firmy (zaangażowanie, wymiar \textit{oddanie}). Dodatkowo sensowny system promowania pracowników, nagrody i dodatki (satysfakcja, wymiary: \textit{awans},
\textit{nagrody}, \textit{dodatki}) mogą zachęcać pracowników do wkładania większej energii w wykonywane zadania oraz dodatkowo motywować do pokonywania przeszkód (zaangażowanie, wymiar \textit{wigor}).

W związku z tym pierwszym postawionym problemem badawczym jest pytanie, czy istnieje zależność między satysfakcja z pracy, a zaangażowaniem w pracę dla pracowników sektora IT? Jeżeli tak, to między jakimi wymiarami istnieje najsilniejsza zależność, a między jakimi -- najsłabsza.

\subsection{Satysfakcja z pracy wśród sektora IT, a normy dla Polaków}
\label{sec:hip-sat-norms}
Pracownicy sektora IT są to zazwyczaj ludzie dobrze wykształceni z wysoką średnią płacy w porównaniu z resztą Polaków (TODO ref). Firmy informatycznie prężnie rozwijają się na rynku pracy i jest to sektor, który w ostatnim czasie intensywnie się rozwija. Co za tym idzie, firmy informatyczne posiadają kapitał, aby poprawiać różne aspekty pracy, np.: fundować pracownikom karnety na siłownię, dodatkowe ubezpieczenie zdrowotne, inwestować w sprzęt dla pracowników, nagradzać
finansowo. Rozwój takich firm oznacza też możliwość szybszego awansu w szybko rozrastających się firmach (np.: Allegro w Poznaniu). Wszystkie te dane wskazują na uprzywilejowaną pozycję informatyków na rynku pracy w porównaniu z innymi grupami zawodowymi w Polsce. Czy jednak za tym idzie zwiększona satysfakcja z pracy? Może ludzie przyzwyczajają się do lepszych warunków w pracy po kilku latach i przestają je doceniać? A może firmy informatyczne w Polsce nie
przenoszą dobrych wzorców z Zachodu i nie inwestują w systemy motywacyjne pozapłacowe oraz nie interesuje ich jak polepszyć warunki pracy wśród swoich pracowników? Pierwszym krokiem do odpowiedzi na te pytania, będzie sprawdzenie, czy faktycznie informatycy są bardziej zadowoleni ze swojej pracy niż reszta Polaków. Posłużą do tego normy przygotowane przez Instytut Psychologii Wydziału Nauk Społecznych Uniwersytetu Adama Mickiewicza (\ref{sec:app-jss-norms}).

\subsection{Zaangażowanie w pracę wśród sektora IT, a normy dla Polaków}
Rynek pracy dla sektora IT jest bardzo chłonny. Codziennie ukazuje się kilka, jak nie kilkanaście ofert pracy z całej Polski (szczególnie z regionu Dolnego Śląska). Tak ukształtowany rynek pracy sprzyja znajdywaniu pracy najbardziej dopasowanej do danej osoby, tzn. pracy, w której zadania odpowiadają zdolnościom, wiedzy i umiejętnościom danej osoby. Odpowiednia praca sprzyja wigorowi w pracy, absorpcji w zadania oraz oddaniu, czyli wszystkim aspektom zaangażowania. Dodatkowo kształcenie
się w kierunku informatyka wymaga włożenia dużej ilości pracy, aby posiąść umiejętności atrakcyjne z punktu widzenia pracodawcy. Co więcej, dziedzina ta intensywnie się rozwija i wymaga ciągłego dokształcania, aby nie wypaść z rynku. Ze względu na powyższe cechy można posunąć się do konkluzji, że osoby zajmujące się tą działką zazwyczaj nie są przypadkowe, posiadają motywację wewnętrzną i możemy mówić o odpowiedniości pracy w ich przypadku. Czy w związku z tym informatycy są bardziej zaangażowani w swoją pracę
niż ogół Polaków? Do zweryfikowania tej hipotezy wykorzystane zostaną normy przygotowane przez Instytut Psychologii Wydziału Nauk Społecznych Uniwersytetu Adama Mickiewicza (\ref{sec:app-uwes-norms}). 
