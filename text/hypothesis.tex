\chapter{Problemy badawcze oraz stawiane hipotezy}
\section{Zależność między satysfakcją, a zaangażowaniem}
\label{sec:hypothesis-relation}
Intuicyjnym wydaje się istnienie zależności między satysfakcją z pracy, a zaangażowaniem pracowników w wykonywane zadania. Skoro satysfakcja wskazuje na poziom zadowolenia z różnych aspektów pracy (warunków, komunikacji, pensji, sposobu zarządzania, itp.), to takie środowisko pracy, które jest satysfakcjonujące powinno sprzyjać także wysokiemu zaangażowaniu. 

Co więcej wybrane wymiary satysfakcji możemy spróbować teoretycznie przełożyć na wymiary zaangażowania. Skoro organizacja pracy (satysfakcja, wymiar \textit{organizacja pracy}) jest odpowiednia, oznacza to, że jest bardzo mało czynników, które powinny nam przeszkadzać w skupieniu
się całkowitym nad zadaniami (zaangażowanie, wymiar \textit{absorpcji}). Natomiast jeżeli przełożeni dbają o swoich pracowników (satysfakcja, wymiar \textit{nadzór}) oraz cele w firmie są jasno stawiane (satysfakcja, wymiar \textit{komunikacja}), to pracownicy powinny odczuwać sensowność wykonywanych zadań oraz dumę z pracy dla danej firmy (zaangażowanie, wymiar \textit{oddanie}). Dodatkowo sensowny system promowania pracowników, nagrody i dodatki (satysfakcja, wymiary: \textit{awans},
\textit{wyrażanie uznania}, \textit{świadczenia pracownicze}) mogą zachęcać pracowników do wkładania większej energii w wykonywane zadania oraz dodatkowo motywować do pokonywania przeszkód (zaangażowanie, wymiar \textit{wigor}).

W związku z tym pierwszym postawionym problemem badawczym jest pytanie, czy istnieje zależność między satysfakcja z pracy, a zaangażowaniem w pracę dla pracowników sektora IT? Jeżeli tak, to między jakimi wymiarami istnieje najsilniejsza zależność, a między jakimi -- najsłabsza.

\section{Satysfakcja z pracy wśród pracowników sektora IT, a polskie normy}
\label{sec:hip-sat-norms}
Pracownicy sektora IT są to zazwyczaj ludzie dobrze wykształceni z wysoką średnią płacy w porównaniu z resztą Polaków, średnio $1500$ zł więcej niż przeciętni Polacy \cite{web:earnings-it,web:earnings-pl}. Firmy informatycznie prężnie rozwijają się na rynku pracy i jest to sektor, który w ostatnim czasie intensywnie się rozrasta. Co za tym idzie, firmy informatyczne posiadają kapitał, aby poprawiać różne aspekty pracy, np.: fundować pracownikom karnety na siłownię, dodatkowe ubezpieczenie zdrowotne, inwestować w sprzęt dla pracowników, nagradzać
finansowo. Rozwój takich firm oznacza też możliwość szybszego awansu w szybko rozrastających się firmach. Wszystkie te dane wskazują na uprzywilejowaną pozycję informatyków na rynku pracy w porównaniu z innymi grupami zawodowymi w Polsce. Czy jednak za tym idzie zwiększona satysfakcja z pracy? Może ludzie przyzwyczajają się do lepszych warunków w pracy po kilku latach i przestają je doceniać? A może firmy informatyczne w Polsce nie
przenoszą dobrych wzorców z Zachodu i nie inwestują w pozapłacowe systemy motywacyjne oraz nie interesuje ich jak polepszyć organizację pracy swoich pracowników? Pierwszym krokiem do odpowiedzi na te pytania, będzie sprawdzenie, czy faktycznie informatycy są bardziej zadowoleni ze swojej pracy niż reszta Polaków. Posłużą do tego normy przygotowane przez
\href{http://www.psychologia.amu.edu.pl/ip-uam/struktura-zatrudnienia-w-instytucie/curriculum-vitae-teresa-chirkowska-smolak/}{Teresę Chirkowską-Smolak} z \href{http://www.psychologia.amu.edu.pl/}{Instytutu Psychologii Wydziału Nauk Społecznych Uniwersytetu Adama Mickiewicza} (\ref{sec:app-jss-norms}).

\section{Zaangażowanie w pracę wśród pracowników sektora IT, a polskie normy}
Rynek pracy dla sektora IT jest bardzo chłonny. Codziennie ukazuje się kilka, jak nie kilkanaście ofert pracy z całej Polski (szczególnie z regionu Dolnego Śląska). Tak ukształtowany rynek pracy sprzyja znajdywaniu pracy najbardziej dopasowanej do danej osoby, tzn. pracy, w której zadania odpowiadają zdolnościom, wiedzy i umiejętnościom danej osoby. Odpowiednia praca sprzyja wigorowi w pracy, absorpcji podczas wykonywania zadania oraz oddaniu, czyli wszystkim aspektom zaangażowania. Dodatkowo kształcenie
się w kierunku informatyki wymaga włożenia dużej ilości pracy, aby posiąść umiejętności atrakcyjne z punktu widzenia pracodawcy. Co więcej, dziedzina ta intensywnie się rozwija i wymaga ciągłego dokształcania, aby nie wypaść z rynku pracy. Ze względu na powyższe cechy można posunąć się do konkluzji, że osoby zajmujące się tą dziedziną zazwyczaj nie są przypadkowe, posiadają motywację wewnętrzną i możemy mówić o odpowiedniości pracy w ich przypadku. Zgodnie z teorią Kahna (patrz
rozdział \ref{sec:theory-eng-kahn}) powinna istnieć co najmniej zgodność poznawcza. Natomiast zgodnie z teorią Maslacha i in. (patrz rozdział \ref{sec:model-maslach}) oraz Schaufelliego i in. (patrz rozdział \ref{sec:model-schauffeli}) osoby takie prawdopodobnie mają więcej energii i wigoru, tak aby móc się dokształcać i rozwijać. Czy w związku z tym informatycy są bardziej zaangażowani w swoją pracę
niż ogół Polaków? Do zweryfikowania tej hipotezy wykorzystane zostaną normy przygotowane przez \href{http://www.psychologia.amu.edu.pl/ip-uam/struktura-zatrudnienia-w-instytucie/curriculum-vitae-teresa-chirkowska-smolak/}{Teresę Chirkowską-Smolak} z \href{http://www.psychologia.amu.edu.pl/}{Instytutu Psychologii Wydziału Nauk Społecznych Uniwersytetu Adama Mickiewicza} (\ref{sec:app-uwes-norms}). 
