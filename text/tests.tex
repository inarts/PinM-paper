\section{Wykorzystane narzędzia psychologiczne}
\subsection{Job Satisfaction Survey - Paul E. Spector}
\emph{Job Satisfaction Survey} jest profesjonalnym testem psychologicznym składającym się z 36 stwierdzeń: po 4 stwierdzenia na każdy z przyjętych 9 wymiarów satysfakcji z pracy (m.in.: związanych z nastawieniem pracownika do pracy czy organizacją pracy). Wymiary zostały dobrane na podstawie dostępnej literatury na ten temat. Twórcom zależało, aby skale odzwierciedlały najważniejsze aspekty satysfakcji z pracy. Przy czym skale miały być jasno
rozróżnialne (jasne granice). Prezentowane stwierdzenia są przygotowane w obu kierunkach: pozytywnym oraz negatywnym, co ma istotne znaczenie przy liczeniu wyników. 

Zastosowane skale to:
\begin{itemize}
\item Płaca -- płaca za wykonaną pracę
\item Awans -- możliwości awansu
\item Nadzór -- ocena bezpośredniego kierownictwa
\item Świadczenia pracownicze -- premie i dodatki (finansowe i niefinansowe)
\item Warunkowe nagrody -- docenienie, uznanie oraz nagrody za dobrą pracę
\item Organizacja pracy -- obowiązujące zasady i procedury, polityka w firmie
\item Współpracownicy -- osoby, z którymi współpracujesz na co dzień
\item Praca -- natura oraz specyfika wykonywanej pracy
\item Komunikacja - komunikacja w ramach organizacji
\end{itemize}

Pierowtnie \emph{JSS} został stworzony dla pracowników sektora uslugowego, jednak może być stosowany do wszystkich typów organizacji. Obecnie istnieją normy dla wielu typów organizacji oraz pracy, dostępnych na stronie TODO.

Do każdego z 36 stwierdzeń responded ustosunkowuje się wskazując stopień zgodności na sześciostopniowej skali nominalnej:
\begin{itemize}
\item bardzo się zgadzam
\item umiarkowanie się nie zgadzam
\item minimalnie się nie zgadzam
\item minimalnie się zgadzam
\item umiarkowanie się zgadzam
\item bardzo się nie zgadzam
\end{itemize}
Skala ta zwyczajowo jest transofrmowana na skalę liczbową od 1 (,,bardzo się nie zgadzam") do 6 (,,bardzo się zgadzam"). Dzięki takiemu przekodowaniu możliwe są dalsze wyliczenia statystyk. Oczywiście dla stwierdzeń negatywnych wymagane jest dodatkowe przekodowanie w celu uzyskania sumarycznego wyniku dla pozytywnej notacji. Przekodowanie to jest odwróceniem skali liczbowej, czyli:
\begin{itemize}
\item $ 1 -> 6 $
\item $ 2 -> 5 $
\item $ 3 -> 4 $
\item $ 4 -> 3 $
\item $ 5 -> 2 $
\item $ 6 -> 1 $
\end{itemize}

Końcowa satysfkacja pracownika wyliczona za pomocą \emph{JSS} reprezentowana jest w postaci liczbowej na skali ciągłej rozciągającej się od 36 włącznie do 216 włącznie, czyli od zdecydowanego niezadowolenia do zdecydowanego zadowolenia z pracy. Nie ma z góry określonych liczbowych wartości granicznych pomiędzy zadowoleniem, a niezadowoleniem z pracy. Ze względu na konieczność oceny, przyjęto 2 podejśćia:
\begin{itemize}
\item podejście normatywne -- porównanie z normami dla danej populacji,
\item podejście absolutne -- podzielenie skali na 3 przedziały liczbowe, oznaczające po kolei niezadowolenie z pracy, uczucia ambwiwaletne w stosunku do pracy oraz niezadowolenie z pracy.
\end{itemize}

\subsubsection{Podejście normatywne}
Podejście normatywne porównuje badaną próbę osób z normami dla tej samej populacji. Część norm jest ogólniedostępna, w szczególoności dla angielskojęzycznej populacji, na strone twórcy \emph{JSS} TODO. Wykorzystujemy test statystyczny t-studenta podczas porównania stwierdzamy, czy istnieje znacząca statystycznie różnica między badaną grupą, a populacją. W wyniku możemy stwierdzić, że badani są tak samo, bardziej lub mniej zadowoleni z pracy niż przeciętne osoby. Niestety podejście to
ma trzy poważne wady:
\begin{enumerate}
\item Obecnie mały procent typów organizacji i zawodów posiada normy. Wyklucza to możliwość zastosowania tego podejśćia w dużej ilości orgaqnizacji oraz ogranicza możliwości porównania.
\item Normy prezentowane na stronie \emph{JSS} zazwyczaj nie są reprezentatywne, a jedynie stworzone są na podstawie próbek podesłanych autorówi \emph{JSS} przez innych naukowców. W związku z tym nie tworzą ważnych naukowo norm i wyciąganie wniosków na ich podstawie powinno być ostrożne.
\item Większość norm została zebrana z terenów Ameryki Północnej. Satysfakcja z pracy różni się ze względu na kraj, a tym bardziej kulturę, dlatego wykorzystanie obecnych norm w celach innych niż porównanie z krajami Ameryki Północnej jest niemiarodajne.
\end{enumerate}
 
\subsubsection{Podejście absolutne}

Ze względu na wymienione wady podejścia normatywnego, wyznaczono także podejście absolutne bazujące na podziale domeny wyników. Każde ze stwierdzeń jest oceniane na skali od 1 (zdecydowanie się nie zgadzam) do 6 (zdecydowanie się zgadzam). Po uwzględnieniu odwrócenia skali dla stwierdzeń negatywnych, można przedstawić domenę:
\begin{itemize}
\item dla każdej z podskal -- [4, 24],
\item dla ogólnego wyniku -- [36, 216],
\end{itemize}
Domeny te zostały podzielone ze względu na trzy wyróżnione stany:
\begin{description}
\item[niezadowolenie z pracy] \hfill \\
dla każdej z podskal [4, 12), \hfill \\ dla sumarycznego wyniku [36, 108),
\item[ambwiwalentny stosunek do pracy] \hfill \\
dla każdej z podskal [12, 16), \hfill \\ dla sumarycznego wyniku [108, 144],
\item[zadowlenie z pracy] \hfill \\
dla każdej z podskal [16, 24], \hfill \\ dla sumarycznego wyniku [144, 216].
\end{description}

Do powyższych obliczeń można także wykorzystać średnią wartość dla uzyskanych wyników. Wówczas odpowiednio:
\begin{description}
\item[niezadowolenie z pracy] \hfill \\ poniżej 3 ($śr. <3$),
\item[ambwiwalentny stosunek do pracy] \hfill \\ pomiędzy 3 włącznie, a 4 wyłącznie ($ 3<= śr. < 4 $)
\item[zadowlenie z pracy] \hfill \\ powyżej 4 włącznie ($ śr. >= 4 $).
\end{description}

Podejściu to także unika jasnego podziału między zadowoleniem, a niezadowleniem wprowadzając przdział związany z mieszanymi uczuciami odnośnie pracy.

\subsubsection{Powiązania z różnymi aspektami pracy}
Opracowując \emph{JSS} wykonano także inne znane testy psychologiczne (np.: Organizational Commitment Questionnaire -- Mowday, Steers \& Porter, 1979 lub Leader Behavior Leader Behaviour Description Questionnaire -Stogdill, 1963) , aby sprawdzić powiązania wyników \emph{JSS} z innymi aspektami pracy. Najsilniejsza korelacja wyszła z odbiorem pracy i przełożonych, intencjami odejścia oraz zaangażowaniem organizacyjnym.
Podczas badań nad \emph{JSS}em sprawdzono jak wyniki testu korelują z innymi aspektami pracy
\subsection{Ultrecht Work Engagement Scale}
\emph{Ultrecht Work Engagement Scale} jest profesjonalnym testem psychologicznym składającym się z 17 pytań: po 5 (oddanie) lub 6 (absorpcja, wigor) stwierdzeń na każdy z przyjętych wymiarów zaangażowania w pracę. Wymiary zostały stworzone na podstawie konstruktu autorów testu (Schaufeli \& Bakker 2001) na temat zaangażwania (patrz TODO). Co jest istotne, autorzy stworzyli konstrukt niezależnie od definicji wypalenia zawodowego, tworząc w ten sposób model możliwy do
badania zależności między zaangażowaniem, a wypaleniem zawodowym. Prezentowane stwierdzenia dla każdego z wymiarów mają kierunek pozytywny, co ułatwia końcowe liczenie wyników.

Zastosowane skale to:
\begin{itemize}
\item wigor -- energia, odporność psychiczna i upór podczas wykonywania zadań,
\item absorpcja -- pozytywny emocjonalnie stan, kiedy osoba jest pochłonięta zadaniem, a czas ucieka nie wiadomo kiedy,
\item oddanie -- poczucie istotności wykonywanych zadań, duma z pracy.
\end{itemize}

\emph{UWES} jest darmowym testem, dla którego istnieją kilka norm międzynarodowych, w tym holenderska, kanadyjska, czy południowo-amerykańska. 

Do każdego z 17 stwierdzeń respondent ustosunkowywał się na siedmiostopniowej skali intensywności występowania zjawiska:
\begin{itemize}
\item nigdy -- nigdy się nie zdarza,
\item prawie nigdy -- kilka razy w roku ub rzadziej,
\item rzadko -- raz w miesiącu lub rzadziej,
\item czasami -- kilka razy w miesiącu,
\item często -- raz w tygodniu,
\item bardzo często -- kilka razy w tygodniu,
\item zawsze -- każdego dnia.
\end{itemize}

Końcowe zaangażowanie w pracy wyliczane jest za pomocą \emph{UWES} w postaci liczbowej i reprezentowane na skali ciągłej od 0 włącznie do 102 włącznie, czyli od braku zaangażowania, do zaangażowanie pojawiającego się codziennie w pracy. Podobnie jak z \emph{JSS} nie ma ostrej granicy między brakiem zaangażowania, a zaangażowaniem. Przejście między tymi stanami jest płynne. Ze względu sposób odpowiedzi respondentów oraz metodę wyliczania końcowej wartości, przyjęto 2
podejścia:
\begin{itemize}
\item podejście normatywne -- porównanie wyników z już istniejącymi normami dla danej populacji,
\item podejście częstotliwościowe -- określenie stopnia zaangażowania poprzez średnią częstość jego występowania.
\end{itemize}
\subsubsection{Podejście normatywne}
Podejście to jest analogiczne jak już opisane dla \emph{JSS}'a (TODO ref). Dla przypomnienia, polega na porównaniu wyników dla naszej badanej próby z normami wyznaczonymi dla populacji poprzez autorów testu lub innych badaczy. Część z nich jest dostępna na stronie jednego z twórców testu -- Schaufelliego (TODO URL). W wyniku wykonania testu hipotezy statystycznej odnośnie średniej, stwierdzamy, czy istnieje statystycznie istotna różnica między badaną grupą, a grupą, która stanowi
dla nas punkt odniesienia. Za pomocą tej metody możemy stwierdzić, że nasza grupa jest mniej, tak samo lub bardziej zaangażowana niż grupa reprezentatywna. Podejście to ma te same wady, co przy \emph{JSS}ie:
\begin{enumerate}
\item Obecnie istnieje niewiele norm dla krajów, typów zawodów czy typów organizacji. Ogranicza to możliwości wykorzystania tego podejścia tylko do grup posiadających normy.
\item Normy te nie są stworzone dla całej populacji i istnieją uzasadnione wątpliwości co do ich reprezentatywności. Były nadsyłane przez naukowców na całym świecie, na podstawie własnych badań, które mogą posiadać pewne obciążenie odnośnie metody przeprowadzenia badań, w tym metody dotarcia do grupy badanej. Dlatego należy z ostrożnością podchodzić do otrzymanych wyników, jeżeli nie zna się dokładnie historii stworzenia normy.
\item Ze względu na różnice kulturowe normy z innych krajów często są niemiarodajne, jeżeli chodzi o daleko wysunięte wyciąganie wniosków. Jedyne do czego można się posunąć, to proste stwierdzenie, że w danym kraju dana grupa badana jest tak samo, mniej lub bardziej zaangażowana. Hipotezy odnośnie przyczyny powinny być poddane dalszej weryfikacji.
\end{enumerate}

\subsubsection{Podejście częstotliwościowe}
Ze względu na wady podejścia normatywnego, można zastosować podejście częstotliwościowe, bazujące na średniej z wszystkich odpowiedzi respondentów. Wówczas średnią częstotliwość odczuwania zaangażowania w pracy określamy następująco:
\begin{itemize}
\item $[0, 0.99)$ -- raz w roku lub rzadziej,
\item $[1, 1.99)$ -- przynajmniej raz w roku,
\item $[2, 2.99)$ -- przynajmniej raz w miesiącu,
\item $[3, 3.99)$ -- przynajmniej kilka razy w miesiącu,
\item $[4, 4.99)$ -- przynajmniej raz w tygodniu,
\item $[5, 6]$ -- przynajmniej kilka razy w tygodniu lub codziennie. 
\end{itemize}

Jeżeli zastosujemy sumy, wówczas przedziały zmieniają się odpowiednio:
\begin{itemize}
\item $[0, 16]$ -- raz w roku lub rzadziej,
\item $[17, 33]$ -- przynajmniej raz w roku,
\item $[34, 50]$ -- przynajmniej raz w miesiącu,
\item $[51, 67]$ -- przynajmniej kilka razy w miesiącu,
\item $[68, 84]$ -- przynajmniej raz w tygodniu,
\item $[85, 102]$ -- przynajmniej kilka razy w tygodniu lub codziennie. 
\end{itemize}
