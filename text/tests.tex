\section{Wykorzystane narzędzia badawcze}
\subsection{\emph{Job Satisfaction Survey}}
\emph{Job Satisfaction Survey} jest profesjonalnym testem psychologicznym składającym się z 36 stwierdzeń: po 4 stwierdzenia na każdy z przyjętych 9 wymiarów satysfakcji z pracy (m.in.: związanych z nastawieniem pracownika do pracy czy organizacją pracy). Wymiary zostały dobrane na podstawie dostępnej literatury na ten temat. Twórcom zależało, aby skale odzwierciedlały najważniejsze aspekty satysfakcji z pracy. Przy czym skale miały być jasno
rozróżnialne (jasne granice między pojęciami). Prezentowane stwierdzenia są przygotowane w obu kierunkach: pozytywnym oraz negatywnym, co ma istotne znaczenie przy liczeniu wyników. 

Zastosowane skale to:
\begin{itemize}
\item płaca -- płaca za wykonaną pracę;
\item awans -- możliwości awansu;
\item nadzór -- ocena bezpośredniego kierownictwa;
\item świadczenia pracownicze -- premie i dodatki (finansowe i pozafinansowe);
\item warunkowe nagrody -- docenienie, uznanie oraz nagrody za dobrą pracę;
\item organizacja pracy -- obowiązujące zasady i procedury, polityka w firmie;
\item współpracownicy -- osoby, z którymi współpracuje dany pracownik na co dzień;
\item wykonywana praca -- natura oraz specyfika wykonywanej pracy;
\item komunikacja - jasność stawianych celów i zadań na różnym szczeblu w ramach danej organizacji.
\end{itemize}

Pierwotnie \emph{JSS} został stworzony dla pracowników sektora usługowego, jednak może być stosowany do wszystkich typów organizacji. Obecnie istnieją normy dla wielu typów organizacji oraz pracy, dostępnych na stronie autora testu \citep{web:jss-norms}.  

Do każdego z 36 stwierdzeń respondent ustosunkowuje się wskazując stopień zgodności na sześciostopniowej skali Likerta: 
\begin{itemize}
\item ,,bardzo się nie zgadzam'',
\item ,,umiarkowanie się nie zgadzam'',
\item ,,minimalnie się nie zgadzam'',
\item ,,minimalnie się zgadzam'',
\item ,,umiarkowanie się zgadzam'',
\item ,,bardzo się zgadzam''.
\end{itemize}
Skala ta zwyczajowo jest transformowana na skalę liczbową od 1 (,,bardzo się nie zgadzam") do 6 (,,bardzo się zgadzam"). Dzięki takiemu przekodowaniu możliwe jest dalsze opracowanie statystyk (np.: opisowych). Oczywiście dla stwierdzeń skierowanych negatywnie wymagane jest dodatkowe przekodowanie w celu ostatecznego uzyskania sumarycznego wyniku przy założeniu, że wszystkie pytania w kwestionariuszu mają kierunek pozytywny. Przekodowanie to jest odwróceniem skali liczbowej, czyli:

\begin{itemize}
\item $ 1 \rightarrow 6 $
\item $ 2 \rightarrow 5 $
\item $ 3 \rightarrow 4 $
\item $ 4 \rightarrow 3 $
\item $ 5 \rightarrow 2 $
\item $ 6 \rightarrow 1 $
\end{itemize}

Końcowa satysfakcja pracownika wyliczona za pomocą \emph{JSS} reprezentowana jest w postaci liczbowej na skali ciągłej rozciągającej się od 36 włącznie do 216 włącznie, czyli odpowiednio od zdecydowanego niezadowolenia do zdecydowanego zadowolenia z pracy. Nie ma z góry określonych liczbowych wartości granicznych pomiędzy zadowoleniem, a niezadowoleniem z pracy. Ze względu na konieczność końcowej oceny, przyjęto dwa podejścia:
\begin{itemize}
\item podejście normatywne -- porównanie z normami dla danej populacji,
\item podejście absolutne -- podzielenie skali na trzy przedziały liczbowe, oznaczające po kolei niezadowolenie z pracy, uczucia ambiwalentne w stosunku do pracy oraz niezadowolenie z pracy.
\end{itemize}

\subsubsection{Podejście normatywne}
\label{sec:jss-calc-norm}
Podejście normatywne porównuje badaną próbę osób z normami dla tej samej, szerszej lub innej populacji (np.: pracownicy sektora IT są podgrupą wszystkich Polaków). Część norm jest ogólnie dostępna, w szczególności dla angielskojęzycznej populacji, na stronie twórcy \emph{JSS} \citep{web:jss-norms}. W tym wypadku wykorzystujemy test statystyczny t-Studenta (mniejsze grupy) lub statystykę Z dla dwóch zmiennych niezależnych (większe grupy). Na podstawie wyliczonych statystyk
stwierdzamy, czy istnieje znacząca statystycznie różnica między badaną grupą, a populacją. W wyniku możemy stwierdzić, że badani są tak samo, bardziej lub mniej zadowoleni z pracy niż osoby z norm (zależnie od postawionych hipotez). Niestety podejście to
ma trzy poważne wady.
\begin{enumerate}
\item Obecnie mały procent typów organizacji i zawodów posiada normy. Wyklucza to możliwość zastosowania tego podejścia w dużej ilości organizacji i zawodów.
\item Normy prezentowane na stronie \emph{JSS} zazwyczaj nie są reprezentatywne, a jedynie stworzone są na podstawie próbek podesłanych autorowi \emph{JSS} przez innych naukowców. W związku z tym często nie tworzą ważnych naukowo norm i wyciąganie wniosków na ich podstawie powinno być ostrożne.
\item Większość norm została zebrana z terenów Ameryki Północnej. Satysfakcja z pracy różni się ze względu na kraj, a tym bardziej kulturę, dlatego wykorzystanie obecnie istniejących norm w celach innych niż porównanie z krajami Ameryki Północnej będzie niemiarodajne.
\end{enumerate}
 
\subsubsection{Podejście absolutne}

Ze względu na wymienione wady podejścia normatywnego, wyznaczono także podejście absolutne bazujące na podziale domeny wyników. Każde ze stwierdzeń jest oceniane na skali od 1 (,,zdecydowanie się nie zgadzam'') do 6 (,,zdecydowanie się zgadzam''). Po uwzględnieniu odwrócenia skali dla stwierdzeń skierowanych negatywnie, można przedstawić domenę:
\begin{itemize}
\item dla każdej z podskal -- [4, 24],
\item dla ogólnego wyniku -- [36, 216].
\end{itemize}
Domeny te zostały podzielone ze względu na trzy wyróżnione stany:
\begin{description}
\item[niezadowolenie z pracy] \hfill \\
dla każdej z podskal [4, 12), \hfill \\ dla sumarycznego wyniku [36, 108),
\item[ambiwalentny stosunek do pracy] \hfill \\
dla każdej z podskal [12, 16), \hfill \\ dla sumarycznego wyniku [108, 144),
\item[zadowolenie z pracy] \hfill \\
dla każdej z podskal [16, 24], \hfill \\ dla sumarycznego wyniku [144, 216].
\end{description}

Do powyższych obliczeń można także wykorzystać średnią wartość dla uzyskanych wyników. Wówczas odpowiednio:
\begin{description}
\item[niezadowolenie z pracy] średnia poniżej 3,
\item[ambiwalentny stosunek do pracy] pomiędzy 3 włącznie, a 4 wyłącznie,
\item[zadowolenie z pracy] powyżej 4 włącznie.
\end{description}

Warto podkreślić, że podejście to także unika jasnego podziału między zadowoleniem, a niezadowoleniem wprowadzając przedział określający mieszane uczucia odnośnie pracy.

\subsection{\emph{Utrecht Work Engagement Scale}}
\label{sec:tests-eng}
\emph{Utrecht Work Engagement Scale} jest profesjonalnym testem psychologicznym składającym się z 17 pytań: po 5 (wymiar \textit{oddanie}) lub 6 (wymiary \textit{absorpcja}, \textit{wigor}) stwierdzeń na każdy z przyjętych wymiarów zaangażowania w pracę. Wymiary zostały stworzone na podstawie konstruktu autorów testu \emph{UWES} na temat zaangażowania (patrz rozdział \ref{sec:model-schauffeli}) \citep{schaufeli2001burnout}. Co jest istotne, autorzy stworzyli konstrukt niezależnie od definicji wypalenia zawodowego, tworząc w ten sposób model możliwy do
badania zależności między zaangażowaniem, a wypaleniem zawodowym (jednak nie jest to tematem niniejszej pracy). Prezentowane stwierdzenia dla każdego z wymiarów mają kierunek pozytywny, co upraszcza końcowe liczenie wyników.

Zastosowane skale to:
\begin{itemize}
\item wigor -- energia, odporność psychiczna i upór podczas wykonywania zadań;
\item absorpcja -- pozytywny emocjonalnie stan, kiedy osoba jest pochłonięta zadaniem, a czas mija bardzo szybko, w niezauważalny sposób dla badanego;
\item oddanie -- poczucie istotności oraz sensowności wykonywanych zadań, duma z pracy.
\end{itemize}

\emph{UWES} jest darmowym testem, dla którego istnieje kilka norm międzynarodowych, w tym holenderska, kanadyjska, czy południowo-amerykańska. 

Do każdego z 17 stwierdzeń respondent ustosunkowuje się na siedmiostopniowej skali intensywności występowania zjawiska:
\begin{itemize}
\item nigdy -- nigdy się nie zdarza,
\item prawie nigdy -- kilka razy w roku lub rzadziej,
\item rzadko -- raz w miesiącu lub rzadziej,
\item czasami -- kilka razy w miesiącu,
\item często -- raz w tygodniu,
\item bardzo często -- kilka razy w tygodniu,
\item zawsze -- każdego dnia.
\end{itemize}

Końcowe zaangażowanie w pracy wyliczane jest za pomocą przemapowania odpowiedzi do postaci liczbowej:
\begin{itemize}
\item nigdy -- 0,
\item prawie nigdy -- 1,
\item rzadko -- 2,
\item czasami -- 3,
\item często -- 4,
\item bardzo często -- 5,
\item zawsze -- 6.
\end{itemize}

Ostateczny wynik reprezentowany jest na skali ciągłej od 0 włącznie do 6 włącznie (określany na podstawie średniej), czyli od braku zaangażowania, do zaangażowanie pojawiającego się kilka razy w tygodniu lub codziennie w pracy. Konkretne sumaryczne wartości liczbowe przekodowuje się następująco do częstotliwości występowania: 
\begin{itemize}
  \item $[0, 1)$ -- raz do roku lub rzadziej,
  \item $[1, 2)$ -- co najmniej raz w roku,
  \item $[2, 3)$ -- co najmniej raz w miesiącu,
  \item $[3, 4)$ -- co najmniej kilka razy w miesiącu,
  \item $[4, 5)$ -- co najmniej raz w tygodniu,
  \item $[5, 6]$ -- kilka razy w tygodniu lub codziennie.
\end{itemize}

Podobnie jak z \emph{JSS} nie ma klarownej granicy między brakiem zaangażowania, a zaangażowaniem. Przejście między tymi skrajnymi stanami jest płynne. 

Ze względu na grupę badanych respondentów (typ organizacji czy wykonywany zawód) oraz metodę wyliczania końcowej wartości, przyjęto dwa podejścia:
\begin{itemize}
\item podejście normatywne -- porównanie wyników z już istniejącymi normami dla określonej populacji,
\item podejście częstotliwościowe -- określenie stopnia zaangażowania poprzez średnią częstość jego występowania.
\end{itemize}
\subsubsection{Podejście normatywne}
Podejście to jest analogiczne jak już opisane dla \emph{JSS}'a (rozdział \ref{sec:jss-calc-norm}). Dla przypomnienia, polega na porównaniu wyników badanej próby z normami wyznaczonymi dla pewnej określonej populacji. Część z norm jest dostępna na stronie jednego z twórców testu -- Schaufeliego \citep{web:uwes-norms}. W wyniku wykonania testu hipotezy statystycznej odnośnie średniej, stwierdzamy, czy istnieje statystycznie istotna różnica między badaną grupą, a grupą, która stanowi
dla nas punkt odniesienia. Za pomocą tej metody możemy stwierdzić, że nasza grupa jest mniej, tak samo lub bardziej zaangażowana niż grupa reprezentatywna. Podejście to ma te same wady, co przy \emph{JSS}ie:
\begin{enumerate}
\item Obecnie istnieje niewiele norm dla krajów, typów zawodów czy typów organizacji. Ogranicza to możliwości wykorzystania tego podejścia tylko do grup posiadających normy.
\item Normy te nie są stworzone dla całej populacji i istnieją uzasadnione wątpliwości co do ich reprezentatywności. Były nadsyłane przez naukowców na całym świecie, na podstawie własnych badań, które mogą posiadać pewne obciążenie odnośnie metody przeprowadzenia badań, w tym metody dotarcia do grupy badanej. Dlatego należy z ostrożnością podchodzić do otrzymanych wyników, jeżeli nie zna się dokładnie historii stworzenia normy.
\item Ze względu na różnice kulturowe normy z innych krajów często są niemiarodajne, jeżeli chodzi o daleko wysunięte wyciąganie wniosków. Jedyne do czego można się posunąć, to proste stwierdzenie, że w danym kraju dana grupa badana jest tak samo, mniej lub bardziej zaangażowana. Hipotezy odnośnie przyczyny powinny być poddane dalszej weryfikacji.
\end{enumerate}

\subsubsection{Podejście częstotliwościowe}
Ze względu na wady podejścia normatywnego, można zastosować podejście częstotliwościowe, bazujące na średniej z wszystkich odpowiedzi respondentów. Wówczas średnią częstotliwość odczuwania zaangażowania w pracy określamy na skali od ,,raz w roku lub rzadziej'' do ,,przynajmniej kilka razy w tygodniu lub codziennie'' (dokładne przedziały opisano w ramach rozdziału \ref{sec:tests-eng}).

Jeżeli zastosujemy sumy, wówczas przedziały zmieniają się odpowiednio:
\begin{itemize}
\item $[0, 16]$ -- raz do roku lub rzadziej,
\item $[17, 33]$ -- przynajmniej raz w roku,
\item $[34, 50]$ -- przynajmniej raz w miesiącu,
\item $[51, 67]$ -- przynajmniej kilka razy w miesiącu,
\item $[68, 84]$ -- przynajmniej raz w tygodniu,
\item $[85, 102]$ -- przynajmniej kilka razy w tygodniu lub codziennie. 
\end{itemize}
