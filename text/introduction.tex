\chapter{Cel pracy}
Zarządzanie zespołem nie jest trywialnym zadaniem. W ramach pracy z zespołem należy dopilnować poprawnego wykonania założonych celów, a także zarządzać ludźmi. Nie oznacza to tylko wyznaczania im zadań i pilnowania, aby wykonali je w określonym terminie. Zarządzanie zespołem ludzi to nie tyko zarządzanie zadaniami, ale także realizacja potrzeb pracowników.

Na skuteczność zespołu wpływa stan psychiczny poszczególnych pracowników, nie tylko ich umiejętności. Za skutecznością idzie poprawność i terminowość zleconych zadań. Dlatego też tak istotne jest rozeznanie w sprawie przyczyn danego działania pracowników -- zarówno czynów odbieranych negatywnie (wyeliminowanie), jak i pozytywnie (podtrzymanie, wzmacnianie). Tą drugą częścią będzie zajmować się niniejsza praca -- przyczynami powodującymi dobre wykonywanie zadań, w określonych terminach, na
oczekiwanym poziomie (efekt zaangażowania w pracę). Jakie czynniki są istotne dla pracownika badanego sektora, a jakie są pomijalne? 

Ponadto będą badane przyczyny, które powodują, że pracownik chce w danej firmie zostać, jego podejście buduje dobrą atmosferę w pracy oraz pozytywnie wpływa na innych współpracowników (efekt satysfakcji z pracy). Należy pamiętać, że pracownicy niezadowoleni, jasno dający wyraz swojej frustracji, mogą zepsuć nastrój w pracy i przelać swoje niezadowolenie na innych. Dodatkowo nieusatysfakcjonowany pracownik ma większą motywację do odejścia, co może być szkodliwe dla firmy, szczególnie gdy taka osoba ma odpowiednie doświadczenie, wiedzę oraz umiejętności; jest zaufanym, sprawdzonym pracownikiem. Znalezienie nowej osoby na miejsce takiego pracownika zabiera czas, wymaga
wyznaczenia osoby do jego wyszkolenia i obserwacji w celu zbadanie jego
potencjalnych umiejętności. To wszystko wpływa na długość procesu podczas którego staje się przydatnym pracownikiem firmy.

Podsumowując, niniejsza praca ma za zadanie sprawdzić jak w Polsce, a dokładniej w Poznaniu, dba się o pracowników sektora IT. W szczególności badana jest satysfakcja z pracy oraz zaangażowanie w wykonywane zadania. Czy badana grupa zawodowa różni się od średniej Polaków, czy nie? Jaka jest charakterystyka tej grup i jak wiąże się z teorią odnośnie zaangażowania w pracę i satysfakcji z pracy?
