\section{Satysfakcja z pracy i jej wymiary}
\label{sec:theory-sat}
Zanim przejdziemy do pełnej definicji satysfakcji z pracy warto jasno określić czym jest satysfakcja. Wg Słownika języka polskiego PWN \cite{web:pwn-sat}
\begin{quote}
satysfakcja -- uczucie przyjemności i zadowolenia z czegoś
\end{quote}
Idąc tym tropem można przekształcić powyższą definicję na potrzeby tej pracy
\begin{quote}
satysfakcja z pracy -- uczucie przyjemności i zadowolenia z pracy
\end{quote}

Natomiast pełna definicja wg podręcznika psychologii \cite{SchultzSat} jest bardziej złożona i definiuje satysfakcje z pracy jako wielowymiarowy konstrukt, którego elementy składowe wciąż są odkrywane.

\begin{quotation}
Satysfakcję z pracy stanowią pozytywne i negatywne uczucia oraz postawy wobec naszej pracy. Zależy ona od wielu czynników związanych z pracą, poczynając od miejsca parkowania samochodu do poczucia samospełnienia przy realizacji codziennych zadań. Na satysfakcję z pracy mogą również wpływać czynniki indywidualne. Należą do nich wiek, stan zdrowia, staż pracy, stabilność emocjonalna, status społeczny, ulubione rozrywki czy też posiadanie rodziny i kontaktów społecznych. Nasza motywacja i
aspiracje, oraz sposób ich zaspokajania przez pracę, także wpływają na postawy wobec pracy.
\end{quotation}
Wśród innych, niewymienionych czynników wpływających na satysfakcję, wyróżnia się także:
\begin{itemize}
\item styl zarządzania,
\item kulturę w organizacji,
\item autonomię pracy,
\item nagrody i kary,
\item godziny pracy,
\item organizację pracy.
\end{itemize}
Czynniki te mogą być zależne od kraju i jego kultury, jak i od typu organizacji, czy typu wykonywanej pracy, a także zajmowanego stanowiska. Dla przykładu, dla pracownika przy taśmie autonomia w pracy będzie nieistotna, w przeciwieństwie do kadry kierowniczej, dla której to swoboda działania może być istotnym czynnikiem wpływającym na ich wydajność. Idąc dalej, dla pracownika w Stanach Zjednoczonych, gdzie nie ma darmowej służby zdrowia, dodatki, takie jak ubezpieczenie zdrowotne, są bardziej istotne niż w Polsce, gdzie prawo do opieki zdrowotnej wpisane jest w konstytucję. 

Należy tutaj podkreślić, że satysfakcja z pracy nie jest tym samym co motywacja do pracy, czy zaangażowanie w pracę. Intuicyjnie jednak, definicje te są ze sobą powiązane.

\subsection{Modele}
Poniżej zaprezentowany jest krótki przegląd teorii związanych z satysfakcją z pracy.

\subsubsection{Teoria dwuczynnikowa (teoria motywacji-higieny) -- Fredereick Herzberg (1974)}
\label{sec:theory-sat-herz}
Teoria ta wskazuje dwa czynniki, które osobno wpływają na satysfakcje i brak satysfakcji; odpowiednio: motywacja oraz czynniki higieny \cite{herzberg1974motivation}. 

\paragraph{Motywacja.} Wyższa motywacja powoduje wzrost zadowolenia z pracy. Może być widziana jako czynniki wewnętrzne pchające pracownika do osiągnięcia prywatnych i organizacyjnych celów. Motywatorami są takie elementy pracy, które powodują chęć do działania i zapewniają uczucie zadowolenia w pracy:
\begin{itemize}
\item osiągnięcia w pracy, np.: zdobycie kolejnego klienta, zakończenie projektu sukcesem,
\item docenienie, np.: bonus za wydajność, pochwała na zebraniu pracowników,
\item możliwości promocji, np.: awans po roku pracy, szkolenia na wyższe stanowiska.
\end{itemize}
Motywatory te muszą być związane z wykonywaną pracą.

\paragraph{Higiena.} Natomiast czynniki higieny to czynniki związane ze środowiskiem pracy, takie jak:
\begin{itemize}
\item płaca -- wysokość wynagrodzenie w stosunku do włożonego wysiłku i wykonywanych zadań, 
\item nadzór -- kompetencje merytoryczne i sposób zarządzania,
\item organizacja w firmie -- sposób organizacji pracy ułatwiający wykonywanie zadań, 
\item niezależność -- możliwość wykonywania zadań w wybranych przez pracownika sposób,
\item docenienie -- pochwały za dobrze wykonaną pracę,
\item możliwość wykorzystania swoich zdolności,
\item moralność -- praca zgodna z wartościami moralnymi pracownika,
\item nagrody -- niezwiązane z pensją formy uznania: finansowe i pozafinansowe.
\end{itemize}

Teoria ta wzbudza kontrowersje, ze względu na nie branie pod uwagę cech indywidualnych pracowników, a co za tym idzie, te same reakcje na zmiany w motywatorach i czynnikach higieny przez każdego z pracowników. W teorii tej brakuje także wskazanej metody badania obu czynników, co powoduje, że jest ciężka do potwierdzenia empirycznie.


\subsubsection{Teoria wpływu -- Edwin A. Locke (1976)}
\label{sec:theory-sat-locke}

Teoria \cite{locke1976nature} opiera się na różnicy między oczekiwaniami pracownika, a rzeczywistością zastaną w pracy. Satysfakcja z pracy jest odwrotnie proporcjonalna do wielkości tej różnicy. Dla przykładu, optymalna godzina przybycia do pracy dla Anny to 9 rano. Natomiast pracodawca wymusza przybycie o 7 rano. Zgodnie z tą teorią zmniejsza to poziom satysfakcji z pracy u Anny. Z drugiej strony, jeżeli Annie by pozwolono na przychodzenie o 9 wpłynęłoby to na zwiększenie jej zadowolenia.

Dodatkowo teoria ta rozróżnia wagi pomiędzy różnymi wymiarami satysfakcji z pracy, tzn. każdy wymiar posiada inny, zależny od indywidualnych cech pracownika wpływ na zadowolenie. Obrazując to na wcześniej podanym przykładzie, w przeciwieństwie do Anny, dla Tomka godzina przyjścia do pracy ma minimalne znaczenie i w niewielkim stopniu wpływa na jego satysfakcję. Natomiast duże znaczenie dla niego ma werbalne docenienie ze strony kierownictwa. Z czego
wynika, że ze względu na poziom satysfakcji z pracy, Anna potrzebuje kierownika akceptującego elastyczne godziny pracy, natomiast Tomek -- osobę doceniającą jawnie dokonania swoich podwładnych.

Teoria wpływu jest to obecnie najbardziej znany model satysfakcji z pracy.

\subsubsection{Model charakterystyki pracy -- Hackam i Oldham (1976)}
Teoria ta \cite{hackman1976motivation} wskazuje na pięć głównych cech pracy:
\begin{itemize}
\item różnorodność wymaganych umiejętności,
\item identyfikacja z zadaniem (dopasowanie zadań),
\item znaczenie zadania,
\item autonomia,
\item informacja zwrotna,
\end{itemize}
które wpływają na trzy stany emocjonalne
\begin{itemize}
\item poczucie sensowności,
\item odpowiedzialność,
\item znajomość wyników pracy,
\end{itemize}
które z kolei wpływają na efekty pracy, w tym na satysfakcję z pracy. 

Ze względu na szerszy charakter tej teorii, wykorzystywana jest ona do badania charakterystyki pracy (zawierającej poza zadowoleniem z pracy, takie elementy jak absencja czy motywacja do pracy), a nie tylko satysfakcji.

\subsubsection{Teoria dyspozycyjności -- Timothy A. Judge (1998)}
Teoria ta wychodzi z ogólnego założenie, że każdy człowiek ma swoje własne, wewnętrzne predyspozycje, które definiują u niego pewien określony poziom satysfakcji z pracy, niezależnie od wykonywanego zawodu czy zajmowanego stanowiska. Dla takich osób zewnętrzne czynniki, takie jak organizacja pracy, styl zarządzania czy inne, nie mają wpływu na satysfakcję.

Teoria ta jest zgodna z wynikami badań, że poziom zadowolenia z pracy ma tendencje do utrzymywania się na stałym poziomie w czasie oraz między stanowiskami, a także firmami w wypadku wybranych osób. Przykładem takich badań, są badania nad bliźniakami jednojajowymi, które wskazują właśnie na wpływ genów na poziomie 30-40\%  na zadowolenia z pracy. Jednak teoria ta jest krytykowana i istnieją już wyniki badań wskazujące na brak zależności \cite{schultz1985family}.

Szczególnym zawężeniem tej teorii, jest teoria \emph{Core Self-evaluations Model} zaproponowana przez Timothy A. Judge w 1998 \cite{judge1998dispositional,judge2001relationship}. Wg niej istnieją 4 podstawowe elementy samooceny związane z satysfakcją z pracy:
\begin{itemize}
\item poczucie własnej wartości -- wprost proporcjonalne,
\item ocena własnych umiejętności -- wprost proporcjonalne,
\item umiejscowienie kontroli -- wprost proporcjonalne do wewnętrznego poczucia kontroli (my wpływamy na sytuacje),
\item neurotyczność -- odwrotnie proporcjonalne.
\end{itemize}


\subsection{Sposoby pomiaru}
Satysfakcja z pracy mierzona jest zazwyczaj przy pomocy kwestionariuszy, które badają ogólny poziom zadowolenia lub wybrane aspekty pracy związane z satysfakcją. Ankiety takie są rozsyłane lub rozdawane pracownikom, którzy są proszeni o anonimowe jej uzupełnienie. Dodatkowo badania są zazwyczaj dobrowolne. Powoduje to, że testy wypełniane są przez podzbiór wszystkich pracowników o nieznanej charakterystyce, np.: tylko najbardziej zaangażowani wypełniają testy. W
związku z tym ciężko stwierdzić jak zebrane wyniki odzwierciedlają rzeczywisty poziom zadowolenia wśród całej populacji badanych.

Przykładowym kwestionariuszem satysfakcji z pracy jest Minnesocki Kwestionariusz Satysfakcji (\emph{MSQ}) \cite{weiss1967manual}, który mierzy 20 właściwości pracy, w tym: osiągnięcia, autonomię, warunki pracy, uznanie, kreatywność, czy kompetencje kierownictwa. Składa się on ze 100 pytań, po 5 na każdy z wymiarów. Można także zmierzyć
ogólną satysfakcję sumując wyniki na każdym z wymiarów. Przy czym zastosowana skala to 5-stopniowa skala Likerta, która pozwala na zmierzenie intensywności poziomu satysfakcji.

Poza kwestionariuszami wykorzystywane są także (ale rzadziej):
\begin{itemize}
\item wywiady -- pracownicy oceniają wybrane aspekty swojej pracy podczas rozmowy z osobą wykonującą badania; pozwala na interakcję między badanym, a badaczem, a co za tym idzie doprecyzowanie na bieżąco wybranych kwestii;
\item test niedokończonych zdań -- pracownicy kończą zdania w dowolny sposób (np.: ,,Moja praca jest..."), co zostawia swobodę odpowiedzi (pytania otwarte) jednak wymaga więcej pracy od badacza podczas przygotowania wyników do analizy (interpretacja oraz pogrupowanie odpowiedzi);
\item technika incydentów krytycznych -- opisanie sytuacji w pracy, które wiązały się ze szczególną satysfakcja z pracy lub jej zdecydowanym brakiem; wadą jest badanie tylko skrajnych stanów emocjonalnych.
\end{itemize}

Badając ogólną satysfakcję z pracy, otrzymujemy informacje na temat poziomu zadowolenia pracownika z pracy. Niestety tracimy informacje na temat składowych tego odczucia. Dla przykładu, pracownik może być bardzo zadowolony ze swojej pensji, natomiast wyjątkowo negatywnie podchodzić do organizacji pracy. Wówczas otrzymujemy średnik wynik zadowolenia, który nie jesteśmy w stanie zinterpretować.

Z drugiej strony badając poszczególne wymiary satysfakcji z pracy, możemy stracić ogólny poziom zadowolenia. Praca składa się z tak wielu aspektów, że ciężko określić je wszystkie i umieścić w teście, tak aby odzwierciedlały ogólny poziom satysfakcji. Badania przeprowadzone przez Scarpello i Campell w 1983 \cite{scarpello1983job} wykazały, że korelacja między krótką wersją \emph{MSQ} (po 1 pytanie na każdy z 20 wymiarów), a ogólną satysfakcją z pracy była niska. Doprowadziło to do odkrycia
kolejnych 
5 wymiarów poprzez przeprowadzenie dodatkowych wywiadów:
\begin{itemize}
\item elastyczny czas pracy,
\item narzędzia i wyposażenia miejsca pracy,
\item zagospodarowanie przestrzeni,
\item ułatwianie pracy przez współpracowników,
\item życzliwość w kontaktach z ludźmi.
\end{itemize}
\subsection{Wpływ cech indywidualnych pracownika}
\label{sec:theory-sat-age}
Poza czynnikami zewnętrznymi, na satysfakcję z pracy wpływają oczywiście cechy indywidualne pracownika. Oczywiście cech tych pracodawca nie jest w stanie zmienić, natomiast podczas interpretacji wyników pomiaru zadowolenia z pracy warto brać pod uwagę wynik dotychczasowych badań na ten temat. Pozwoli to na wyciągnięcie poprawnych wniosków.

\paragraph{Wiek} Im osoba jest starsza tym bardziej jest zadowolona z pracy. 

Jednym z wyjaśnień jest zbyt mało ambitny rodzaj pracy. Młodzi, niedoświadczeni pracownicy dostają zadania poniżej swojej umiejętności. Natomiast starsi mają więcej możliwości wykazania się, większą odpowiedzialność, a co za tym idzie lepsze pensje, nagrody, itd. Inne wyjaśnienie wiąże się z urealnieniem oczekiwań wobec pracy wraz z doświadczeniem (czyli
także wiekiem) i poszukiwaniem brakujących aspektów satysfakcji w innych obszarach życia.
\paragraph{Płeć} Brak różnic między kobietami i mężczyznami w Zachodniej Europie \cite{de1991gender}. Natomiast istnieją badania wskazujące na różnice między kobietami zmuszonymi do pracy poprzez sytuacje finansową, a kobietami nastawionymi na karierę zawodową (czyli pracującymi dobrowolnie). Druga grupa wykazuje większe zadowolenie z pracy.
\paragraph{Zdolności poznawcze} Często praca związana z wyższymi dodatkami, płacą, autonomią itd. związana jest z poziomem inteligencji danej osoby. Jej zdolności poznawcze powinny być wysokie, aby mogła sprostać stawianym zadaniom, a co za tym idzie być zadowolona z wykonywanej pracy i aspektów z nią związanych. Gdy zadanie przerasta daną osobę, czuje się ona źle: nie wykonała zadania, poza tym okazało się, że brak jej pewnych umiejętności.
\paragraph{Doświadczenia zawodowe} Satysfakcja z pracy zmniejsza się wraz z latami w tej samej firmie. 

Związane jest to z zatrzymanie rozwoju zawodowego pracownika z czasem, czyli brakiem poszerzania swoich umiejętności oraz brakiem nowych wyzwań. Dodatkowo zmiana pracy wiąże się z weryfikacją swoich umiejętności, a także nowymi możliwościami awansu. Korelację tą potwierdziły badania na grupie angielskich inżynierów \cite{newton1991further}. Ci, którzy zmieniali częściej prace byli bardziej z niej zadowoleni.
\paragraph{Wykorzystanie umiejętności} Pracownicy są bardziej zadowoleni z pracy, jeżeli wykorzystują w niej jak najwięcej już wcześniej posiadanych umiejętności.
\paragraph{Odpowiedniość pracy} Wg badań Eltona i Smarta osoby, które znalazły pracę dopasowaną do ich zdolności wykazywały większe zadowolenie z pracy niż pozostali \cite{elton1988extrinsic}. Wiąże się to ze sprawniejszym i łatwiejszym wykonywaniem pracy, która jest związana z naszymi umiejętnościami.
\paragraph{Status pracy} Co jest łatwe do przewidzenia, im wyższy status pracy (np.: wysokie stanowisko kierownicze) tym większa satysfakcja z pracy. Wiąże się to na pewno z aspektami pracy, które na takich stanowiskach są bardzo dobrze zaspokajane, np.: autonomia, płaca, nagrody, dodatki.

\subsection{Wpływ satysfakcji na pracowników}
\label{sec:theory-sat-infl}
\paragraph{Produktywność} W niektórych badaniach odkryto słaby wpływ satysfakcji na produktywność w pracy. Przy czym przy ocenie tych badań należy wziąć pod uwagę, że w wypadku części zadań w pracy ciężko ocenić obiektywnie poziom ich wykonania, z oczywistych przykładów: praca lekarza. Lawler i Porter sugerują, że relacja ta ma odwrotny kierunek, tzn. produktywność wpływa na satysfakcję z pracy \cite{lawler1967effect}.
\paragraph{Zachowania prospołeczne i nieproduktywne} Zachowania prospołeczne korelują pozytywnie z satysfakcją z pracy, natomiast niezadowolenie wpływa na zachowania nieproduktywne. Pracownicy zadowoleni są bardziej skorzy do pomocy współpracownikom, lepiej obsługują klientów. W przeciwieństwie do osób niezadowolonych, które wykazują zachowania aspołeczne i szkodzące firmie (np.: nieuprzejme potraktowanie klientów).
\paragraph{Absencja} Steel i Rentsch wykazali związek pomiędzy niską satysfakcją z pracy, a wysoką absencją \cite{steel1995influence}. Natomiast Blau wykazał podobną zależność między satysfakcją z pracy, a spóźnianiem się \cite{blau1994developing}. Osoby o niskim poziomie zadowolenia niechętnie chodzą do pracy, a co za tym idzie są bardziej skłonni nie przychodzić do niej (przy nadarzającej się okazji) lub spóźnić się.  \paragraph{Fluktuacja.} Judge w 1993
wykazał, że wśród pracowników sektora zdrowia istnieje powiązanie między niską satysfakcją z pracy, a odchodzeniem z niej \cite{judge1993does}. Z ciekawych, istotnych faktów, dodatkowo osoby te posiadały wysoką satysfakcję z życia.
\paragraph{Satysfakcja z innych aspektów życia} Wg badań Crohana, Antonucciego, Adlemanna i Colemana satysfakcja z pracy koreluje pozytywnie z satysfakcją z życia codziennego (osobistego i rodzinnego) \cite{crohan1989job}. Dalsze badania wykazały wzajemną zależność satysfakcji z pracy i z życia, z nieznacznie większym wpływem satysfakcji z życia \cite{judge1993another}. 
