\subsection{Zaangażowanie w pracę}
Przed zdefiniowaniem zaangażowania w pracę z punktu widzenia psychologii, warto przytoczyć potoczną definicję tego słowa. Wg Słownikia JP PWN:
\begin{quote}
zaangażować się -- włożyć w coś wiele wysiłku, czasu lub pieniędzy
\end{quote}
Zawężająć powyższą definicję do pracy
\begin{quote}
zaangażować się w pracę -- włożyć w pracę wiele wysiłku, czasu lub pieniędzy
\end{quote}

Ze względu na to, że zaangażowanie w pracę jest dość młodym, a co za tym idzie rzadko póki co badanym konceptem (w przciewieństwie do np.: satysfakcji z pracy), brakuje badań wiążących zaangażowanie w prace z cechami osobowości pracowników.
\subsubsection{Modele}
\begin{description}
\item[Kahn 1990]
Pierwszy model stworzony przez W. Kahna w 1990  roku. Wychodzi z założenia, że zaangażowanie to konstrukt związany z wyrażaniem siebie fizycznie, emocjonalnie i poznawczo podczas wykonywania pracy. W 2004 na podstawie tego modelu May, Gilson \& Harter zoperacjonalizowali ten kosntrukt i opracowali test psychologiczny badający zaangażowanie w pracę.
\item[Maslach, Jackson \& Leiter 1996]
Teoria ta definiuje zaangażowanie jako przeciwieństwo wypalenia zawodowego i tworzy kontinuum między tymi dwoma pojęciami. Osoby wypalone zazwyczaj są wyczerpane i cynicznie podchodzą do pracy, natomiast zaangażowane są pełne energii i entuzjastycznie podchodzą do wykonywanych zadań. Dodatkowo isteniej różnica między wydajnością uzyskaną przy tak samo włożonym wysiłu w pracy. Oczywiście osoby bardziej wypalone osiągają słabsze rezultaty.

Teoria ta spotyka się z głosami krytyki. Istnieją wątpliwości czy faktycznie zaangażowanie można tak jednoznacznie przedstawić jako przeciwieństwo wypalenia zawodowego. Czy na pewno kiedy pracownik nie jest zaangażowany oznacza to wypalenia zawodowego? Intuicyjnie chce się odpowiedzieć nie. Aby móc zbadać zależność faktyczną zależność między zaangażowaniem, a wypaleniem należało stworzyć nowy konstrukt nie opierający się na teorii wypalenia zawodowego.
\item[Schauffeli, Bakker 2001]
\end{description}


\subsubsection{Sposoby pomiaru}
Do badania zaangażowanie w pracę obecnie wykorzystuje się testy psychologiczne.

Na podstawie teorii Maslach'a, Jacksona \& Leitera wykorzystuje się kwstionariusz do badania wypalenia zawodowego -- Maslach Burnout Inventory (\emph{MBI}). Jeżeli osoba uzyska tam ,,złe" wyniki, tzn. mało punktów na skali wypalenia i cynizmu, a dużo na skali wydajności w pracy, wówczas uznaje się, że pracownik jest zaangażowany w pracę.

\subsubsection{Wpływ zaangażowania na pracowników}
